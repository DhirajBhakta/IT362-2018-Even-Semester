\chapter{Introduction}

In the past few years, the video content on the Internet increased tremendously. Advancements in social media (Facebook, Twitter, Instagram etc.) have resulted in an abundance of content supply and demand. It has been observed that the number of people willing to spend more time on a website are relatively very less. Now-a-days related video recommendation is \textit{ubiquitous} and \textit{pervasive} on the Internet. From YouTube statistics it can be inferred that over six billion hours of movies have been watched every month. Furthermore, video upload rate is 100 movies per minute. All the above mentioned reasons led to the necessity of development of sophisticated technology for movie retrieval, discovery and recommendations. Movie suggestions are the movies the user is presented with, in relevance to the current movie being watched ($V_W$).\par
The movies being recommended can be purely content based (tags, genres etc.) or can be collaborative filtering based (used by most of the modern recommender systems). Recommendations via collaborative filtering approach is suitable for movies with good views while the same fails for fresh content and tail videos which have very noisy and sparse co-view data. To overcome sparsity and noise related issues, researchers have combined collaborative filtering with content based models to develop hybrid models. CNN, Netflix, Hulu etc. all use hybrid recommender systems to improve \textit{user's browsing experience} through click-through analytics and co-view data.\par
Usually movie (video) is modeled as a set of content data which includes a set of \textit{data topics} extracted from many sources such as metadata, Wikipedia articles, uploader keywords (frequently used), freebase entities, common queries used to search, names of play-lists. Then, these assigned (weighted) topics are used to obtain a set of related movies (videos) for recommendations and suggestions. Furthermore, to achieve an \textit{effective topic based} movie representation, weights are assigned to each topic tagged with a movie. Two techniques have been presented to associate weights to topics. The first technique involves a well known retrieval method which uses  inverse document frequencies (\textit{idf}) and topic frequencies (\textit{tf}). The second technique involves using the implicit (explicit) user feedback. Now, this feedback can be utilized to learn the optimal weights of topics (supervised learning).\par
The topic based representation of the movie being watched ($V_W$) is a \textit{query} which is addressed to the inverted index, the index saves video representations based on topics as \textit{documents}. This way, the movie suggestion problem can be seen as a \textit{retrieval of movies-over-inverted index of documents}. To find the most optimal movie recommendations (even in large volumes of videos), usage of query optimization algorithms can improve efficiency. The top ranked videos (documents) retrieved as a result to the query ($V_W$) are suggested to the viewer as a set of \textbf{content based} video recommendations which (recommendations) can either be presented to the user without any modifications or are appended to the results of co-view based movie (video) recommender (hybrid modeling) system.
