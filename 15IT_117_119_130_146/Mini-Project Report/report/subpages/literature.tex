\chapter{Literature Review}

\section{Background}

There has been immense progress in the field of video retrieval due to the increasing user base and user demands on video sharing sites like YouTube, dailymotion, etc. This has led to addressing the various drawbacks and challenges to this video recommendation system. 

One of the common challenges includes the position bias which is the case with most retrieval systems. Position bias is the positive bias which is introduced towards the first few retrieved results. Sometimes users even neglect the last couple of videos retrieved by the IR system. This leads to those videos getting a bad feedback recursively and finally the system starts ignoring them. Hence both the system and user relevance is affected. \par
However it was seen that position bias is not the only bias affecting the system. Yisong Yue et. al \cite{yue2010} introduced a new concept of presentation bias. This sort of bias is predominantly seen in all on-line video sharing platforms. Presentation bias is introduced when the one set of videos have title bolding and the other set doesn’t. This prompts the user to select the video from the former set even though their quality is low or they are not related. Title Bolded videos are perceived by the users to be more attractive.\par
The existing video recommendation systems heavily rely on collaborative filtering as a method to suggest videos. These systems are based on co-view data. Both Davidson et. al. \cite{davison2010} and Baluja et. al. \cite{baluja2008} describe such a system. The reason for the success of co view based filtering is because the metadata could be noisy. This is seen in the case when user closes the browser before the system receives the long-watch notification. So, there is inconsistency in the various metrics like abandonment rate. Whereas co-view graphs are nothing but directed graphs depicting a particular user session. Baluja et. al. \cite{baluja2008} introduces a concept of an adsorption algorithm. It’s a learning and classification framework which uses a rich graph structure and labelled objects.\par
Collaborative filtering as a method performs well for popular and frequently watched videos but in the case of fresher and tail content videos there is a massive degradation in performance. The reason for this is that the co-view data is noisy and sparse. Hence this filtering algorithm would not be effective if it’s only based on co-view data.\par
To tackle this some hybrid approaches have been experimented and implemented. Bo Yang \cite{yang2007} presents a hybrid approach based on relevance feedback and multi-modal fusion. Their model is capable of working without any user preference information. This is very different from the traditional approaches. The multi-modal relevance is depicted by aural, video and textual relevance which is used to recommend videos. However their system was only tested for a small portion of TRECVID dataset having preset sample videos. Hence the system would not be favorable for a large collection of videos.\par
The approach given below is a hybrid approach tested for a larger audience and a large movie database. Table \ref{tab:1} summarizes the existing works.

\begin{table}[h]
  \centering
  \bgroup
  \def\arraystretch{1.5}
  \caption{Summary of Existing Works}
  \label{tab:1}
  \begin{tabular}{>{\raggedright}p{2.4cm}>{\raggedright}p{3cm}>{\raggedright}p{4cm}>{\raggedright}p{4cm}} 
    \toprule
    \textbf{Author} & \textbf{Approach} & \textbf{Merits} & \textbf{Limitations} \tabularnewline
    \midrule
     Davidson J. et al. \cite{davison2010} & Collaborative filtering & Recommendations are more accurate & Presentation bias, only co-viewed videos 
 \tabularnewline
     Shumeet Baluja et al. \cite{baluja2008} & Co-view graph & Recommendations are more accurate & Presentation bias, only co-viewed videos \tabularnewline
     Bo Yang et al. \cite{yang2007} & Hybrid model & Considers both video info and co-views (multimodal) & Small scale \tabularnewline
    \bottomrule
  \end{tabular}
  \egroup
\end{table}

\section{Identified Gaps}
This session briefs about the gaps in the existing recommender systems based on the literature survey. Firstly, the existing recommender (suggestion) systems base their existence on \textit{collaborative filtering} methods, which might work for popular, more viewed, more liked, dense data; while the same fails for sparse data and tail (fresh) videos. It can also be seen that there is a lack of \textit{content based} video retrieval systems. Also, \textit{topic based retrieval} has never been done on a large scale.

\section{Problem Statement}
Given a large corpus of movies (here, movie trailers) and appropriate metadata (genre and user submitted tags); propose  a methodology to suggest a list of movies the user would like to watch next based on the current movie.

\section{Objectives}

The main objective of the recommendation system is to suggest to the user what she would enjoy watching next based on the current movie. This is done by representing user feedback in the form of a co-view based graph. In the representation of this co-view based graph, every movie is represented by a \textit{node}. The current node is linked to any other nodes, if they are often watched in the same session. The nodes in the proximity of the query node and related to the topic is suggested to the user. To find the topical relation,a hybrid dataset composed of the \textit{IMDb movie dataset} comprising of last 35 years of movies and their respective genres along with a set of user defined tags is used. 